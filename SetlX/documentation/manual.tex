\def\myDocumentTypeArticle{-}
\InputIfFileExists{\jobname_header}{}{\input{config/header_V001}}

%%%%%%%%%%%%%%%%%%%%%%%%%%%%%% document specific %%%%%%%%%%%%%%%%%%%%%%%%%%%%%%

% define the author, title etc
\globalTitle{Interpreter Manual}
\globalSubject{setlX \input{version}}
\globalAuthor{Herrmann, Tom}
\globalDate{\today}
\globalKeywords{readme}

% frequently used configuration options
\setboolean{linkHighlighting}{true}

%%%%%%%%%%%%%%%%%%%%%%%%%%%%%%%%%%%% setup %%%%%%%%%%%%%%%%%%%%%%%%%%%%%%%%%%%%

\InputIfFileExists{\jobname_configuration}{}{\input{config/configuration_V001}}

%%%%%%%%%%%%%%%%%%%%%%%%%%%%%%%%%%%%%%%%%%%%%%%%%%%%%%%%%%%%%%%%%%%%%%%%%%%%%%%
%%%%%%%%%%%%%%%%%%%%%%%%%%%%%%%% document text %%%%%%%%%%%%%%%%%%%%%%%%%%%%%%%%
%%%%%%%%%%%%%%%%%%%%%%%%%%%%%%%%%%%%%%%%%%%%%%%%%%%%%%%%%%%%%%%%%%%%%%%%%%%%%%%

\begin{document}
\begin{titlepage}
\maketitle
\vfill
\tableofcontents
\end{titlepage}

\section{Overview}

This is the manual of \textbf{setlX}, an interpreter for the \textbf{SetlX} programming-language.

\section{Usage}
    If file paths are supplied as parameters for this program, then they will be
    parsed and executed.

    The interactive mode will be started if called without any file parameters.
    Note that input will be executed after entering a empty line.
    The 'exit;' statement will terminate the interpreter.

\nonSubsection{Additional parameters}
        --help
            displays some helpful information
        --predictableRandom
            always returns the same pseudo random sequence of numbers from the
            internal random number generator (for debugging)
        --real32
        --real64
        --real128
        --real256
            sets the width of the real-type in bits (64 is the sane default)
        --verbose
            display the parsed program before executing it
            (has no effect in interactive mode)
        --version
            displays the interpreter version and terminates

\section{System Requirements}
    - To execute this program the Java JRE Version 1.5 (aka `JRE 5') or higher
      must be installed.
    - To build the interpreter from source the corresponding Java JDK has to be
      present as well.

\section{Build \& Launching}

\subsection{Unix-like OS}
    The interpreter will try to automatically build a self-contained `.jar' file
    on first launch (if no manually build version is already present).

    The program can be launched by changing into the `interpreter' sub-directory

        cd interpreter

    and executing a SetlX program

        ./setlX <path>/<name>.stlx

    or launching the interactive mode

        ./setlX


    Optionally the interpreter can be intelligently (re)build by executing

        make

    and/or

        make jar

    The later will create a self-contained `.jar' file. This file can be
    launched on all Java-Platforms without additional scripts, jars or
    environment-variables by executing

        java -jar setlX.jar <path>/<name>.stlx

    or

        java -jar setlX.jar


    Some additional make targets are also available:

        make interactive

    will build a current version from source and execute the interactive mode;

        make test

    will build a current version from source and execute a small set of tests;

        make clean

    will delete all created binary files, except the `.jar' file;

        make dist-clean

    will delete all created files.

\subsection{Microsoft Windows}
    The interpreter will try to automatically build a self-contained '.jar' file
    on first launch.

    It can be launched by changing into the 'interpreter' sub-directory

        cd interpreter

    and executing a SetlX program

        win-setlX.cmd <path>\\<name>.stlx

    or launching the interactive mode

        win-setlX.cmd


\section{Limitations}
 (compared to other Setl versions)
    - all variables and functions are 'case sensitive'
        predefined functions and keywords use camel case and begin with a lower case letter
    - most explicit type conversions are unsupported
    - ignoring positions in list assignments is not possible
        example: [x,-,y] := [1,2,3]; where '2' would not be assigned to anything
    - explicit constants ('const') are unsupported
    - object oriented functions are not supported

\section{Disclaimer}
    This program uses the antlr parser generator in version 3.3.
    As per the license in antlr/antlr\_LICENSE.txt, antlr is not guaranteed to
    work and might even destroy all life on this planet.


\end{document}
