\chapter{Statistical Distributions}
In recent years, \href{https://en.wikipedia.org/wiki/Data_mining}{data mining} has become ever more
important.  As data mining is closely related to statistics,  \setlx\ provides several statistical
distributions.  This chapter gives an overview over all statistical distributions that are supported. 

\section{Beta Distribution} \label{sec:beta_distribution}
The \href{https://en.wikipedia.org/wiki/Beta_distribution}{beta distribution} has two parameters $\alpha > 0$
and $\beta > 0$ and is defined on the interval $(0,1)$. The probability density function of the beta distribution is defined as  
\\[0.2cm]
\hspace*{1.3cm}
$\ds f(x) := \frac{x^{\alpha-1} \cdot (1-x)^{\beta-1}}{\mathtt{B}(\alpha,\beta)}$.
\\[0.2cm]
Here, $\mathtt{B}$ denotes the \href{https://en.wikipedia.org/wiki/Beta_function}{beta function} that is used to ensure
that the total probability integrates to 1. The beta function is defined as  
\\[0.2cm]
\hspace*{1.3cm}
$\ds \mathtt{B}(\alpha,\beta) := \frac{\Gamma(\alpha) \cdot \Gamma(\beta)}{\Gamma(\alpha + \beta)}$.
\\[0.2cm]
Here $\Gamma$ denotes the \href{https://en.wikipedia.org/wiki/Gamma_function}{gamma function}. This function is defined as
\\[0.2cm]
\hspace*{1.3cm}
$\ds \Gamma(x) := \int\limits_{0}^{\infty } t^{x-1} \cdot e^{-t}\,\mathrm{d}t$.
\\[0.2cm]
The probability
density function of the beta distribution can be computed in \setlx\ 
via the expression
\\[0.2cm]
\hspace*{1.3cm}
\texttt{stat\_beta(x, alpha, beta)}.
\\[0.2cm]
As a convenience, it can be plotted on a previously created \texttt{canvas}\footnote{
  A \texttt{canvas} is created via the command
  ``\texttt{canvas := plot\_createCanvas(\textsl{name})};''.  Here  \textsl{name} is a string that serves as the name of the
  canvas.  Plotting is discussed in more detail in the following chapter.
}
 via the expression
\\[0.2cm]
\hspace*{1.3cm}
\texttt{stat\_beta\_plot(canvas, alpha, beta)}. 
\\[0.2cm]
The cumulative distribution function of the beta distribution is given as
\\[0.2cm]
\hspace*{1.3cm}
$\ds F(x) = \frac{\mathtt{B}(x, \alpha,\beta)}{\mathtt{B}(\alpha,\beta)}$. 
\\[0.2cm]
Here, $\mathtt{B}(x,\alpha,\beta)$ is the 
\href{https://en.wikipedia.org/wiki/Beta_function#Incomplete_beta_function}{incomplete beta function} which is
defined as
\\[0.2cm]
\hspace*{1.3cm}
$\ds \mathtt{B}(x,\alpha,\beta) := \int\limits_{0}^{x}t^{\alpha-1} \cdot (1-t)^{\beta-1}\,\mathrm{d}t$,
\\[0.2cm]
while $\mathtt{B}(\alpha,\beta)$ denotes the beta function defined above.
The cumulative distribution function of the beta distribution is computed via the expression 
\\[0.2cm]
\hspace*{1.3cm}
\texttt{stat\_betaCDF(x, alpha, beta)}.
\\[0.2cm]
As a convenience, it can plotted on a given \texttt{canvas} via the command
\\[0.2cm]
\hspace*{1.3cm}
\texttt{stat\_betaCDF\_plot(canvas, alpha, beta)}. 

\section{Cauchy Distribution}
The \href{https://en.wikipedia.org/wiki/Cauchy_distribution}{Cauchy distribution} has a location parameter $t$
and a scale parameter $s$ where $s > 0$. The probability density function of the Cauchy distribution is defined
as 
\\[0.2cm]
\hspace*{1.3cm}
$\ds f(x) := \frac{1}{\pi} \cdot \frac{s}{s^2 + (x-t)^2}$ \quad for all $x \in \mathbb{R}$.
\\[0.2cm]
The probability density function of the Cauchy distribution is computed in \setlx\ via the expression 
\\[0.2cm]
\hspace*{1.3cm}
\texttt{stat\_cauchy(x, t, s)} 
\\[0.2cm]
As a convenience, it can be plotted on a given \texttt{canvas} via the command
\\[0.2cm]
\hspace*{1.3cm}
\texttt{stat\_cauchy\_plot(canvas, t, s)}.
\\[0.2cm]
The cumulative distribution function of the Cauchy distribution is defined as 
\\[0.2cm]
\hspace*{1.3cm}
$\ds F(x) := \frac{1}{2} + \frac{1}{\pi} \cdot \arctan\left(\frac{x-t}{s}\right)$.
\\[0.2cm]
The cumulative distribution function of the Cauchy distribution is computed via the expression
\\[0.2cm]
\hspace*{1.3cm}
\texttt{stat\_cauchyCDF(x, t, s)}.
\\[0.2cm]
As a convenience, it is plotted on a given \texttt{canvas} by the command
\\[0.2cm]
\hspace*{1.3cm}
\texttt{stat\_cauchyCDF\_plot(canvas, t, s)}. 

\section{Chi-Squared Distribution}
The \href{https://en.wikipedia.org/wiki/Chi-squared_distribution}{chi-squared distribution}
 (also known as $\chi^2$-distribution) has only one parameter $k$, where $k$ is a positive natural number,
 i.e.~$k \in \mathbb{N}$ and $k \geq 1$. The probability density function of the 
chi-squared distribution is defined as 
\\[0.2cm]
\hspace*{1.3cm}
$\ds f(x;\,k) :=  \ds \frac{1}{2^{k/2} \cdot\Gamma(k/2)} \cdot x^{k/2-1} \cdot e^{-x/2}$ \quad for all $x > 0$.
\\[0.3cm]
Here $\Gamma$ denotes the \href{https://en.wikipedia.org/wiki/Gamma_function}{gamma function} that is defined as
\\[0.2cm]
\hspace*{1.3cm}
$\ds \Gamma(x) := \int\limits_{0}^{\infty } t^{x-1} \cdot e^{-t}\,\mathrm{d}t$.
\\[0.2cm]
The probability density function of the chi-squared distribution is computed 
via the expression
\\[0.2cm]
\hspace*{1.3cm}
\texttt{stat\_chiSquared(x, k)}.
\\[0.2cm]
As a convenience, it can be plotted on a given \texttt{canvas} with the command
\\[0.2cm]
\hspace*{1.3cm}
\texttt{stat\_chiSquared\_plot(canvas, k)}. 
\\[0.2cm]
The cumulative distribution function of the chi-squared distribution is defined as
\\[0.2cm]
\hspace*{1.3cm}
$\ds F(x;\,k) := \frac{\gamma(k/2,\,x/2)}{\Gamma(k/2)}$,
\\[0.2cm]
where $\gamma(k/2,\,x/2)$ is the 
\href{https://en.wikipedia.org/wiki/Incomplete_gamma_function}{lower incomplete gamma function}, which is defined as  
\\[0.2cm]
\hspace*{1.3cm}
$\ds \gamma(s, x) := \int\limits_{0}^{x}{t^{s-1} \cdot e^{-t} \mathrm{d}t} $
\\[0.2cm]
The cumulative distribution function of the chi-squared distribution is computed in \setlx\ via the expression
\\[0.2cm]
\hspace*{1.3cm}
\texttt{stat\_chiSquaredCDF(x, k)}.
\\[0.2cm]
As a convenience, it can be plotted on a given \texttt{canvas} with the command
\\[0.2cm]
\hspace*{1.3cm}
\texttt{stat\_chiSquaredCDF\_plot(canvas, k)}.

\section{Exponential Distribution}
The \href{https://en.wikipedia.org/wiki/Exponential_distribution}{exponential distribution} is given as
\\[0.2cm]
\hspace*{1.3cm}
$\ds f(x;\lambda) := \lambda \cdot e^{-\lambda \cdot x}$ \quad for $x \geq 0$.
\\[0.2cm]
Here, the parameter $\lambda$ is a scale parameter that is bigger than zero, i.e.~we have $\lambda > 0$.
The probability density function of the exponential distribution is computed via the expression
\\[0.2cm]
\hspace*{1.3cm}
\texttt{stat\_exponential(x, l)}.
\\[0.2cm]
As a convenience, it can be plotted on a given \texttt{canvas} with the command
\\[0.2cm]
\hspace*{1.3cm}
\texttt{stat\_exponential\_plot(canvas, l)}. 
\\[0.2cm]
The cumulative distribution function is given as
\\[0.2cm]
\hspace*{1.3cm}
$\ds F(x;\lambda) = 1 - e^{-\lambda \cdot x}$ \quad for $x \geq 0$.
\\[0.2cm]
The cumulative distribution function of the exponential distribution is computed via the expression
\\[0.2cm]
\hspace*{1.3cm}
\texttt{stat\_exponentialCDF(x, l)}.
\\[0.2cm]
As a convenience, it can be plotted on a given \texttt{canvas} with the command
\\[0.2cm]
\hspace*{1.3cm}
\texttt{stat\_exponentialCDF\_plot(canvas, l)}. 

\section{Fisher-Snedecor Distribution} 
The
\href{https://en.wikipedia.org/wiki/F-distribution}{Fisher-Snedecor distribution},
or F-distribution, has the two parameters $a$ and $b$ which are positive real numbers. 
If $a$ and $b$ are natural numbers, then they are also known as the \emph{degrees of freedom}.  The probability density function is
defined as 
\\[0.2cm]
\hspace*{1.3cm}
$\ds f(x) := \frac{1}{\mathtt{B}(a/2,b/2)} \cdot \left(\frac{a}{b}\right)^{a/2} \cdot x^{a/2 - 1} \cdot \left(1+\frac{a}{b} \cdot x\right)^{-(a+b)/2}$
\\[0.2cm]
where $\mathtt{B}$ denotes the \href{https://en.wikipedia.org/wiki/Beta_function}{beta function}. The beta function is defined as  
\\[0.2cm]
\hspace*{1.3cm}
$\ds \mathtt{B}(\alpha,\beta) := \frac{\Gamma(\alpha) \cdot \Gamma(\beta)}{\Gamma(\alpha + \beta)}$.
\\[0.2cm]
Here $\Gamma$ denotes the \href{https://en.wikipedia.org/wiki/Gamma_function}{gamma function}, which is defined as
\\[0.2cm]
\hspace*{1.3cm}
$\ds \Gamma(x) := \int\limits_{0}^{\infty } t^{x-1} \cdot e^{-t}\,\mathrm{d}t$.
\\[0.2cm]
The probability density function of the F-distribution is computed by the expression
\\[0.2cm]
\hspace*{1.3cm}
\texttt{stat\_fisher(x, a, b)}.
\\[0.2cm]
As a convenience, it can be plotted on a given \texttt{canvas} with the command
\\[0.2cm]
\hspace*{1.3cm}
\texttt{stat\_fisher\_plot(canvas, a, b)}. 
\\[0.2cm]
The cumulative distribution function is defined as
\\[0.2cm]
\hspace*{1.3cm}
$\ds F(x, a,b) := \frac{1}{\mathtt{B}\left(\frac{a}{2}, \frac{b}{2}\right)} \cdot
  \mathtt{B}\left(\frac{a \cdot x}{a \cdot x + b},\frac{a}{2}, \frac{b}{2} \right)
$
\\[0.2cm]
where $\mathtt{B}(x,a,b)$ denotes the \href{https://en.wikipedia.org/wiki/Beta_function#Incomplete_beta_function}{incomplete beta function}
that is defined as
\\[0.2cm]
\hspace*{1.3cm}
$\ds \mathtt{B}(x,a,b) := \int\limits_{0}^{x}t^{a-1} \cdot (1-t)^{b-1}\,\mathrm{d}t$,
\\[0.2cm]
while $\mathtt{B}(a,b)$ denotes the \href{https://en.wikipedia.org/wiki/Beta_function}{beta function}, which has been 
defined at the beginning of this subsection.
The cumulative distribution function of the F-distribution is computed by the expression
\\[0.2cm]
\hspace*{1.3cm}
\texttt{stat\_fisherCDF(x, a, b)}.
\\[0.2cm]
As a convenience, it can be plotted on a given \texttt{canvas} via the command
\\[0.2cm]
\hspace*{1.3cm}
 \texttt{stat\_fisherCDF\_plot(canvas, a, b)}. 


\section{Gamma Distribution}\label{sec:gamma_distribution}
The \href{https://en.wikipedia.org/wiki/Gamma_distribution}{gamma distribution} has the shape parameter $\alpha$ and the scale 
parameter $\lambda$. Both of these parameters are positive real numbers, i.e.~we have
\\[0.2cm]
\hspace*{1.3cm}
$\alpha \in \mathbb{R}$, \quad $\lambda \in \mathbb{R}$, \quad $\alpha > 0$, \quad and \quad $\lambda > 0$.
\\[0.2cm]
The probability density function is defined as 
\\[0.2cm]
\hspace*{1.3cm}
$\ds f(x;\alpha,\lambda) := \frac{\lambda}{\Gamma(\alpha)} \cdot (\lambda \cdot x)^{\alpha-1} \cdot e^{-\lambda \cdot x}$,
\\[0.3cm]
where $\Gamma$ is the \href{https://en.wikipedia.org/wiki/Gamma_function}{gamma function}, which is defined as
\\[0.2cm]
\hspace*{1.3cm}
$\ds \Gamma(x) := \int\limits_{0}^{\infty } t^{x-1} \cdot e^{-t}\,\mathrm{d}t$.
\\[0.2cm]
The probability density function of the gamma distribution is computed via the expression
\\[0.2cm]
\hspace*{1.3cm}
\texttt{stat\_gamma(x, alpha, 1/lambda)}.
\\[0.2cm]
As a convenience, it can be plotted on a given \texttt{canvas} via the command
\\[0.2cm]
\hspace*{1.3cm}
\texttt{stat\_gamma\_plot(canvas, alpha, 1/lambda)}. 
\\[0.2cm]
The cumulative distribution function is given as
\\[0.2cm]
\hspace*{1.3cm}
$\ds F(x; \alpha, \lambda) = \frac{1}{\Gamma(\alpha)} \cdot \gamma\left(\alpha, \lambda \cdot x\right)$
\\[0.2cm]
where $\gamma$ denotes the
\href{https://en.wikipedia.org/wiki/Incomplete_gamma_function}{lower incomplete gamma function}, which is defined as
\\[0.2cm]
\hspace*{1.3cm}
$\ds \gamma(\alpha, x) := \int\limits_{0}^{x}{t^{\alpha-1} \cdot e^{-t} \mathrm{d}t} $.
\\[0.2cm]
The cumulative distribution function of
the gamma distribution is computed in \setlx\ via the expression 
\\[0.2cm]
\hspace*{1.3cm}
\texttt{stat\_gammaCDF(x, alpha, 1/lambda)}.
\\[0.2cm]
As a convenience it can be plotted on a given \texttt{canvas} via the command 
\\[0.2cm]
\hspace*{1.3cm}
\texttt{stat\_gammaCDF\_plot(canvas, alpha, 1/lambda)}. 

\section{L�vy Distribution}
The \href{https://en.wikipedia.org/wiki/Levy_distribution}{L�vy distribution} has two parameters:
\begin{enumerate}[(a)]
\item $\mu$ is a real number and is called the \emph{location parameter} \ and
\item $c$ is a positive real number that serves as the \emph{scale parameter}. 
\end{enumerate}
The distribution is defined for all $x \in \mathbb{R}$ such that $x \geq \mu$. 
The probability density function of the L�vy distribution is defined as 
\\[0.2cm]
\hspace*{1.3cm}
$\ds f(x;\mu,c) := \sqrt{\frac{c}{2\pi}} \cdot \frac{1}{(x-\mu)^{3/2}} \cdot \exp\left(-\frac{c}{2 \cdot (x-\mu)}\right)$.
\\[0.2cm]
The probability density function of the L�vy distribution is computed via the expression
\\[0.2cm]
\hspace*{1.3cm}
\texttt{stat\_levy(x, mu, c)}. 
\\[0.2cm]
As a convenience, it can be plotted on a given \texttt{canvas} with the command
\\[0.2cm]
\hspace*{1.3cm}
 \texttt{stat\_levy\_plot(canvas, mu, scale)}. 
\\[0.2cm]
The cumulative distribution function of the L�vy distribution satisfies 
\\[0.2cm]
\hspace*{1.3cm}
$\ds F(x;\mu,c)=\textrm{erfc}\left(\sqrt{\frac{c}{2(x-\mu)}}\right)$.
\\[0.2cm]
Here, $\mathrm{erfc}$ is the 
\href{https://en.wikipedia.org/wiki/Error_function#Complementary_error_function}{complementary error function}, which is defined as
the integral
\\[0.2cm]
\hspace*{1.3cm}
$\ds \mathrm{erfc}(x) := \frac{2}{\sqrt{\pi}} \cdot \int\limits_{x}^{\infty} \exp\bigl(-t^{2}\bigr)\,\mathrm{d}t$.
\\[0.2cm]
The cumulative distribution function of the L�vy distribution is computed via the expression
\\[0.2cm]
\hspace*{1.3cm}
 \texttt{stat\_levyCDF(x, mu, c)}. 
\\[0.2cm]
As a convenience, it can be plotted on a given \texttt{canvas} with the command 
\\[0.2cm]
\hspace*{1.3cm}
\texttt{stat\_levyCDF\_plot(canvas, mu, c)}. 

\section{Log-Normal Distribution}
The \href{https://en.wikipedia.org/wiki/Log-normal_distribution}{log-normal distribution} is characterized by two parameters:
\begin{enumerate}[(a)]
\item $\mu$ is the mean of the natural logarithm of the associated random variable.
\item $\sigma$ is the standard deviation of the natural logarithm of the associated random variable.  Of course, the standard variation
      has to be positive, i.e.~$\sigma > 0$.  
\end{enumerate}
The log-normal distribution is defined for all $x \geq 0$. The probability density function of the
log-normal distribution is given as 
\\[0.2cm]
\hspace*{1.3cm}
$\ds f_X(x) := 
 \frac{1}{x} \cdot \frac{1}{\sigma \cdot \sqrt{2\pi\,}} \cdot \exp\left( -\frac{\bigl(\ln(x)-\mu\bigr)^2}{2\cdot\sigma^2} \right)
$.
\\[0.2cm]
The probability density function of the log-normal distribution is computed via the expression
\\[0.2cm]
\hspace*{1.3cm}
\texttt{stat\_logNormal(x, mu, sigma)}.
\\[0.2cm]
As a convenience, it can be plotted on a given \texttt{canvas} with the command
\\[0.2cm]
\hspace*{1.3cm}
\texttt{stat\_logNormal\_plot(canvas, mu, sigma)}. 
\\[0.2cm]
The cumulative distribution function of the log-normal distribution is given as
\\[0.2cm]
\hspace*{1.3cm}
$\ds F_X(x) = \frac{1}{2} \cdot \mathtt{erfc} \left(-\frac{\ln(x) - \mu}{\sigma\cdot\sqrt{2}}\right)$.
\\[0.2cm]
Here $\mathtt{erfc}$ is the 
\href{https://en.wikipedia.org/wiki/Error_function#Complementary_error_function}{complementary error function},
which is defined as the integral
\\[0.2cm]
\hspace*{1.3cm}
$\ds \mathrm{erfc}(x) := \frac{2}{\sqrt{\pi}} \cdot \int\limits_{x}^{\infty} \exp\bigl(-t^{2}\bigr)\,\mathrm{d}t$.
\\[0.2cm]
The cumulative distribution function of the log-normal distribution is computed via the expression
\\[0.2cm]
\hspace*{1.3cm}
\texttt{stat\_logNormalCDF(x, mu, sigma)}.
\\[0.2cm]
As a convenience, it can be plotted on a given \texttt{canvas} with the command
\\[0.2cm]
\hspace*{1.3cm}
\texttt{stat\_logNormalCDF\_plot(canvas, mu, sigma)}. 


\section{Normal Distribution}
The \href{https://en.wikipedia.org/wiki/Normal_distribution}{normal distribution} is characterized by two parameters:
\begin{enumerate}[(a)]
\item $\mu$ is the \emph{location} parameter and is the same as the mean of the associated random variable.
\item $\sigma$ is the \emph{scale} parameter and is identical to the standard deviation of the associated random variable.
      Therefore we must have $\sigma > 0$. 
\end{enumerate}
The probability density function of the normal distribution is defined as 
\\[0.2cm]
\hspace*{1.3cm}
$\ds f(x) := \frac{1}{\sigma \cdot \sqrt{2 \cdot \pi}} \cdot \exp\left(-\frac{(x-\mu)^2}{2 \cdot \sigma^2}\right)$.
\\[0.2cm]
The probability density function of the normal distribution is computed via the expression
\\[0.2cm]
\hspace*{1.3cm}
 \texttt{stat\_normal(x, mu, sigma)}.
\\[0.2cm]
As a convenience, it can be plotted on a given \texttt{canvas} with the command 
\\[0.2cm]
\hspace*{1.3cm}
\texttt{stat\_normal\_plot(canvas, mu, sigma)}. 
\\[0.2cm]
The cumulative distribution function of the normal distribution is given as
\\[0.2cm]
\hspace*{1.3cm}
$\ds F(x) = \frac{1}{2} \cdot \mathtt{erfc}\left(-\frac{x-\mu}{\sigma \cdot\sqrt{2}}\right)$.
\\[0.2cm]
Here $\mathtt{erfc}$ is the 
\href{https://en.wikipedia.org/wiki/Error_function#Complementary_error_function}{complementary error function},
which is defined as the integral
\\[0.2cm]
\hspace*{1.3cm}
$\ds \mathrm{erfc}(x) := \frac{2}{\sqrt{\pi}} \cdot \int\limits_{x}^{\infty} \exp\bigl(-t^{2}\bigr)\,\mathrm{d}t$.
\\[0.2cm]
The cumulative distribution function of the normal distribution is computed via the expression
\\[0.2cm]
\hspace*{1.3cm}
\texttt{stat\_normalCDF(x, mu, sigma)}.
\\[0.2cm]
As a convenience, it can be plotted on a given \texttt{canvas} with the command 
\\[0.2cm]
\hspace*{1.3cm}
\texttt{stat\_normalCDF\_plot(canvas, mu, sigma)}. 

\section{Pareto Distribution}
The \href{https://en.wikipedia.org/wiki/Pareto_distribution}{Pareto distribution} is characterized by two parameters:
\begin{enumerate}[(a)]
\item $\alpha$ is the \emph{shape} parameter and has to be positive.
\item $x_\mathrm{m}$ is the \emph{scale} parameter and also has to be positive.
\end{enumerate}
The probability density function is defined for all $x \geq x_\mathrm{m}$ and is given as 
\\[0.2cm]
\hspace*{1.3cm}
$f(x) := \ds \frac{\alpha \cdot x_{\mathrm{m}}^\alpha}{x^{\alpha+1}}$.
\\[0.2cm]
The probability density function of the Pareto distribution is computed via the expression
\\[0.2cm]
\hspace*{1.3cm}
 \texttt{stat\_pareto(x, xm, alpha)}. 
\\[0.2cm]
As a convenience, it can be plotted on a given \texttt{canvas} with the command 
\\[0.2cm]
\hspace*{1.3cm}
\texttt{stat\_pareto\_plot(canvas, xm, alpha)}. 
\\[0.2cm]
The cumulative distribution function of the Pareto distribution is given as
\\[0.2cm]
\hspace*{1.3cm}
$\ds F(x) = 1 - \left(\frac{x_\mathrm{m}}{x}\right)^\alpha$ \quad for $x \geq x_\mathrm{m}$.
\\[0.2cm]
The cumulative distribution function of the Pareto distribution is computed via the expression 
\\[0.2cm]
\hspace*{1.3cm}
\texttt{stat\_paretoCDF(x, xm, alpha)}.
\\[0.2cm]
As a convenience, it can be plotted on a given \texttt{canvas} with the command 
\\[0.2cm]
\hspace*{1.3cm}
\texttt{stat\_paretoCDF\_plot(canvas, xm, alpha)}. 

\section{Student's $t$-Distribution}
The \href{https://en.wikipedia.org/wiki/Student\%27s_t-distribution}{Student's $t$-distribution} 
is characterized by one parameter $\nu$, which is called the \emph{degrees of freedom}.  The degrees of freedom
is a positive natural number, i.e.~we must have $\nu \in \mathbb{N}$ and $\nu \geq 1$.
The probability density function of the Student's $t$-distribution is defined for all $x \in \mathbb{R}$ as
\\[0.2cm]
\hspace*{1.3cm}
$\ds f(x) := 
\frac{\Gamma(\frac{\nu+1}{2})} {\sqrt{\nu\cdot\pi\;}\cdot\Gamma(\frac{\nu}{2})} \cdot\left(1+\frac{x^2}{\nu}\right)^{\!-\frac{\nu+1}{2}}
$.
\\[0.2cm]
Here $\Gamma$ denotes the \href{https://en.wikipedia.org/wiki/Gamma_function}{gamma function}, which is defined as
\\[0.2cm]
\hspace*{1.3cm}
$\ds \Gamma(x) := \int\limits_{0}^{\infty } t^{x-1} \cdot e^{-t}\,\mathrm{d}t$.
\\[0.2cm]
The probability density function of the Student's $t$-distribution is computed via the expression
\\[0.2cm]
\hspace*{1.3cm}
 \texttt{stat\_student(x, nu)}.
\\[0.2cm]
As a convenience, it can be plotted on a given \texttt{canvas} with the command
\\[0.2cm]
\hspace*{1.3cm}
\texttt{stat\_student\_plot(canvas, nu)}. 
\\[0.2cm]
The cumulative distribution function of the Student distribution has the form
\\[0.2cm]
\hspace*{1.3cm}
$\ds F(x) = 1 - \frac{1}{2 \cdot \mathtt{B}\left(\frac{\nu}{2}, \frac{1}{2}\right)}  
            \cdot \mathtt{B}\left(\frac{\nu}{{x^2+\nu}},\frac{\nu}{2}, \frac{1}{2}\right)
$,
\\[0.2cm]
where $\mathtt{B}(x;\,a,b)$ denotes the \href{https://en.wikipedia.org/wiki/Beta_function#Incomplete_beta_function}{incomplete beta function},
which is defined as
\\[0.2cm]
\hspace*{1.3cm}
$\ds \mathtt{B}(x;\,a,b) := \int\limits_{0}^{x}t^{a-1} \cdot (1-t)^{b-1}\,\mathrm{d}t$,
\\[0.2cm]
while $\mathtt{B}(a,b)$ denotes the \href{https://en.wikipedia.org/wiki/Beta_function}{beta function}. The beta function is defined as  
\\[0.2cm]
\hspace*{1.3cm}
$\ds \mathtt{B}(\alpha,\beta) := \frac{\Gamma(\alpha) \cdot \Gamma(\beta)}{\Gamma(\alpha + \beta)}$.
\\[0.2cm]
The cumulative distribution function of the Student's $t$-distribution is computed via the expression
\\[0.2cm]
\hspace*{1.3cm}
\texttt{stat\_studentCDF(x, nu)}.
\\[0.2cm]
As a convenience, it can be plotted on a given \texttt{canvas} with the command
\\[0.2cm]
\hspace*{1.3cm}
\texttt{stat\_studentCDF\_plot(canvas, nu)}. 

\section{Weibull Distribution}
The \href{https://en.wikipedia.org/wiki/Weibull_distribution}{Weibull distribution} is characterized by two
parameters:
\begin{enumerate}[(a)]
\item $k$ is the \emph{shape} parameter and is a positive real number.
\item $\lambda$ is the \emph{scale} parameter and is a positive real number.
\end{enumerate}
The Weibull distribution is defined for $x \geq 0$. The probability density function of the Weibull 
distribution is defined as 
\\[0.2cm]
\hspace*{1.3cm}
$\ds f(x) := \lambda \cdot k \cdot (\lambda \cdot x)^{k-1} \cdot \exp\bigl(-(\lambda \cdot x)^k\bigr)$.
\\[0.2cm]
The probability density function of the Weibull distribution is computed via the expression
\\[0.2cm]
\hspace*{1.3cm}
\texttt{stat\_weibull(x, k, 1/lambda)} 
\\[0.2cm]
As a convenience, it can be plotted on a given \texttt{canvas} with the command 
\\[0.2cm]
\hspace*{1.3cm}
\texttt{stat\_weibull\_plot(canvas, k, 1/lambda)}. 
\\[0.2cm]
The cumulative distribution function of the Weibull distribution is given as
\\[0.2cm]
\hspace*{1.3cm}
$\ds F(x) = 1 - \exp\bigl(-(\lambda \cdot x)^k\bigr)$
\\[0.2cm]
The cumulative distribution function of the Weibull distribution is computed via the expression
\\[0.2cm]
\hspace*{1.3cm}
\texttt{stat\_weibullCDF(x, k, 1/lambda)} 
\\[0.2cm]
As a convenience, it can be plotted on a given \texttt{canvas} with the command 
\\[0.2cm]
\hspace*{1.3cm}
\texttt{stat\_weibullCDF\_plot(canvas, k, 1/lambda)}. 


		
%%% Local Variables: 
%%% mode: latex
%%% TeX-master: "tutorial"
%%% End: 
