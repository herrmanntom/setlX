% new title etc commands, which set latex, pdf attribute and defines \gtitle etc for later use
\newcommand{\globalTitle}[1]{
	\title{#1}
	\hypersetup{pdftitle={#1}}
	\def\gTitle{#1}
}
\newcommand{\globalSubject}[1]{
	\ifdefined\subject
		\subject{#1}
	\fi
	\hypersetup{pdfsubject={#1}}
	\def\gSubject{#1}
}
\newcommand{\globalAuthor}[1]{
	\author{#1}
	\hypersetup{pdfauthor={#1}}
	\def\gAuthor{#1}
}
\newcommand{\globalDate}[1]{
	\date{#1}
	\def\gDate{#1}
}
\newcommand{\globalKeywords}[1]{
	\hypersetup{pdfkeywords={#1}}
	\def\gKeywords{#1}
}

% chapter/section without number, but listed in toc
\newcommand{\nonChapter}[1]{
	\chapter*{#1}
	\ifthenelse{\boolean{tocAllingnNonChaptersWithNumberedChapters}}{
	  % then
		\addcontentsline{toc}{chapter}{\hspace{1.35em}#1}
	}{
	  % else
		\addcontentsline{toc}{chapter}{#1}
	}
}
\newcommand{\nonSection}[1]{
	\section*{#1}
	\ifthenelse{\boolean{tocAllingnNonSectionsWithNumberedSections}}{
	  % then
	  	\ifdefined\chapter
			\addcontentsline{toc}{section}{---\hspace{1.3em}#1}
		\else
			\addcontentsline{toc}{section}{\hspace{1.35em}#1}
		\fi
	}{
	  % else
	  	\ifdefined\chapter
			\addcontentsline{toc}{section}{--- #1}
		\else
			\addcontentsline{toc}{section}{#1}
		\fi
	}
}
\newcommand{\nonSubsection}[1]{
	\subsection*{#1}
	\ifthenelse{\boolean{tocAllingnNonSectionsWithNumberedSections}}{
	  % then
	  	\ifdefined\chapter
			\addcontentsline{toc}{subsection}{---\hspace{2.2em}#1}
		\else
			\addcontentsline{toc}{subsection}{---\hspace{1.3em}#1}
		\fi
	}{
	  % else
		\addcontentsline{toc}{subsection}{--- #1}
	}
}
\newcommand{\nonSubsubsection}[1]{
	\subsubsection*{#1}
	\ifthenelse{\boolean{tocAllingnNonSectionsWithNumberedSections}}{
	  % then
	  	\ifdefined\chapter
	  		% usually subsubsections will not be displayed in toc if chapters are first level
			\addcontentsline{toc}{subsubsection}{---\hspace{3.1em}#1}
		\else
			\addcontentsline{toc}{subsubsection}{---\hspace{2.2em}#1}
		\fi
	}{
	  % else
		\addcontentsline{toc}{subsubsection}{--- #1}
	}
}

% basic figure/table command, which centers, provides a caption and labels
\newcommand{\basicIncludeStructure}[6]{
	\begin{#1}[#2] % position
		\begin{center}
		{#3} % content
		\caption[#5]{#6\label{#4}} % short caption | caption | label
		\end{center}
	\end{#1}
}

% new figure command, which centers, provides a caption and labels
\newcommand{\includeFigure}[5][h]{
	\basicIncludeStructure
		{figure} % type
		{#1} % position
		{#2} % figure content
		{#3} % label
		{#4} % short caption
		{#5} % caption
}

% new table command, which provides a caption and labels
\newcommand{\includeTable}[5][h]{
	\ifthenelse{\boolean{tablesAsFigures}}{
	  % then
		\def\tableIncludeType{figure}
	}{
	  % else
		\def\tableIncludeType{table}
	}
	\basicIncludeStructure
		{\tableIncludeType}
		{#1} % position
		{#2} % table content
		{#3} % label
		{#4} % short caption
		{#5} % caption
}

% new listing command, which provides a caption and labels
\newcommand{\includeListing}[6][h]{
	\ifthenelse{\boolean{listingsAsFigures}}{
	  % then
		\includeFigure
			[#1] % position
			{ \lstinputlisting[#3]{#2} } % options | path/filename
			{#4} % label
			{#5} % short caption
			{#6} % caption
	}{
	  % else
		\begin{singlespace}
			\hspace{1em}
			\lstinputlisting[#3,caption={[#5]{#6}},label={#4}]{#2} % options | short caption | caption | label | path/filename
		\end{singlespace}
	}
}

%% new picture command, which centers, provides a caption and labels
%\newcommand{\includePicture}[6][h]{
%	\includeFigure
%		[#1] % position
%		{ \includegraphics[#3]{#2} } % options | path/filename
%		{#4} % label
%		{#5} % short caption
%		{#6} % caption
%}

% mail address with reduced margin for error
\newcommand{\mailto}[1]{\mbox{\href{mailto:#1}{\texttt{#1}}}} % teletype style makes it look similar to \url{}

% strikes a line through the argument text
\newlength{\strikeOutLen} % to save length of line
\newcommand{\strikeOut}[1]{
	\settowidth{\strikeOutLen}{#1} % set length of line
	\mbox{#1\protect\hspace{-\strikeOutLen}\rule[0.5ex]{\strikeOutLen}{0.1ex}}
}

% full line \framebox for comments etc
\newcommand{\boxed}[1]{
	\par
	\noindent
	\framebox[\linewidth]{
		\addtolength{\linewidth}{-5mm}
		\parbox{\linewidth}{#1}
		\addtolength{\linewidth}{5mm}
	}
	\par
}

% universal box used for warnings, todos, etc
\newcommand{\universalBox}[2]{\boxed{\textbf{#1:}\begin{center}\texttt{\large{#2}}\end{center}}}
% some sort of todo box
\newcommand{\todo}[1]{\universalBox{ToDo}{#1}}
% some sort of warning box
\newcommand{\warn}[1]{\universalBox{Warning}{#1}}
