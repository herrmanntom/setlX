\documentclass[a4paper,oneside]{scrartcl}
\def\myDocumentType{article}

% conditionals
\usepackage{ifthen}

% enable special German/European characters
%\usepackage[official]{eurosym}
\usepackage{ucs}
\usepackage[utf8x]{inputenc}

%% set available languages
%\usepackage[ngerman,english]{babel}

% enable (code) listings
\usepackage{listings}

%% enable graphics
%\usepackage{graphicx}

% add hyperlinks
\usepackage[
	pdftex, % use driver `close' to pdfLaTeX
	unicode=true, % use unicode strings
	bookmarksopen=true, % expand bookmarks by default
	bookmarksnumbered=true % show chapter numbers in bookmarks
]{hyperref}

% point links to top left corner of respective element, not to caption
\usepackage[all]{hypcap}

% enable bold-face in typewriter
\DeclareFontShape{OT1}{cmtt}{bx}{n}{<5><6><7><8><9><10><10.95><12><14.4><17.28><20.74><24.88>cmttb10}{}

%%prerender some unicode characters, to enable their use in predefined commands like \title etc
%\PrerenderUnicode{ÄäÖöÜüß}

%% same for listings
%\lstset{literate= {Ö}{{\"O}}1 {Ä}{{\"A}}1 {Ü}{{\"U}}1 {ß}{{\ss}}2 {ü}{{\"u}}1 {ä}{{\"a}}1 {ö}{{\"o}}1 }

% booleans for settings
\newboolean{linkHighlighting}
\newboolean{tablesAsFigures}
\newboolean{listingsAsFigures}
\newboolean{tocAllingnNonChaptersWithNumberedChapters}
\newboolean{tocAllingnNonSectionsWithNumberedSections}

% load commands
% new title etc commands, which set latex, pdf attribute and defines \gtitle etc for later use
\newcommand{\globalTitle}[1]{
	\title{#1}
	\hypersetup{pdftitle={#1}}
	\def\gTitle{#1}
}
\newcommand{\globalSubject}[1]{
	\ifdefined\subject
		\subject{#1}
	\fi
	\hypersetup{pdfsubject={#1}}
	\def\gSubject{#1}
}
\newcommand{\globalAuthor}[1]{
	\author{#1}
	\hypersetup{pdfauthor={#1}}
	\def\gAuthor{#1}
}
\newcommand{\globalDate}[1]{
	\date{#1}
	\def\gDate{#1}
}
\newcommand{\globalKeywords}[1]{
	\hypersetup{pdfkeywords={#1}}
	\def\gKeywords{#1}
}

% chapter/section without number, but listed in toc
\newcommand{\nonChapter}[1]{
	\chapter*{#1}
	\ifthenelse{\boolean{tocAllingnNonChaptersWithNumberedChapters}}{
	  % then
		\addcontentsline{toc}{chapter}{\hspace{1.35em}#1}
	}{
	  % else
		\addcontentsline{toc}{chapter}{#1}
	}
}
\newcommand{\nonSection}[1]{
	\section*{#1}
	\ifthenelse{\boolean{tocAllingnNonSectionsWithNumberedSections}}{
	  % then
	  	\ifdefined\chapter
			\addcontentsline{toc}{section}{---\hspace{1.3em}#1}
		\else
			\addcontentsline{toc}{section}{\hspace{1.35em}#1}
		\fi
	}{
	  % else
	  	\ifdefined\chapter
			\addcontentsline{toc}{section}{--- #1}
		\else
			\addcontentsline{toc}{section}{#1}
		\fi
	}
}
\newcommand{\nonSubsection}[1]{
	\subsection*{#1}
	\ifthenelse{\boolean{tocAllingnNonSectionsWithNumberedSections}}{
	  % then
	  	\ifdefined\chapter
			\addcontentsline{toc}{subsection}{---\hspace{2.2em}#1}
		\else
			\addcontentsline{toc}{subsection}{---\hspace{1.3em}#1}
		\fi
	}{
	  % else
		\addcontentsline{toc}{subsection}{--- #1}
	}
}
\newcommand{\nonSubsubsection}[1]{
	\subsubsection*{#1}
	\ifthenelse{\boolean{tocAllingnNonSectionsWithNumberedSections}}{
	  % then
	  	\ifdefined\chapter
	  		% usually subsubsections will not be displayed in toc if chapters are first level
			\addcontentsline{toc}{subsubsection}{---\hspace{3.1em}#1}
		\else
			\addcontentsline{toc}{subsubsection}{---\hspace{2.2em}#1}
		\fi
	}{
	  % else
		\addcontentsline{toc}{subsubsection}{--- #1}
	}
}

% basic figure/table command, which centers, provides a caption and labels
\newcommand{\basicIncludeStructure}[6]{
	\begin{#1}[#2] % position
		\begin{center}
		{#3} % content
		\caption[#5]{#6\label{#4}} % short caption | caption | label
		\end{center}
	\end{#1}
}

% new figure command, which centers, provides a caption and labels
\newcommand{\includeFigure}[5][h]{
	\basicIncludeStructure
		{figure} % type
		{#1} % position
		{#2} % figure content
		{#3} % label
		{#4} % short caption
		{#5} % caption
}

% new table command, which provides a caption and labels
\newcommand{\includeTable}[5][h]{
	\ifthenelse{\boolean{tablesAsFigures}}{
	  % then
		\def\tableIncludeType{figure}
	}{
	  % else
		\def\tableIncludeType{table}
	}
	\basicIncludeStructure
		{\tableIncludeType}
		{#1} % position
		{#2} % table content
		{#3} % label
		{#4} % short caption
		{#5} % caption
}

% new listing command, which provides a caption and labels
\newcommand{\includeListing}[6][h]{
	\ifthenelse{\boolean{listingsAsFigures}}{
	  % then
		\includeFigure
			[#1] % position
			{ \lstinputlisting[#3]{#2} } % options | path/filename
			{#4} % label
			{#5} % short caption
			{#6} % caption
	}{
	  % else
		\begin{singlespace}
			\hspace{1em}
			\lstinputlisting[#3,caption={[#5]{#6}},label={#4}]{#2} % options | short caption | caption | label | path/filename
		\end{singlespace}
	}
}

%% new picture command, which centers, provides a caption and labels
%\newcommand{\includePicture}[6][h]{
%	\includeFigure
%		[#1] % position
%		{ \includegraphics[#3]{#2} } % options | path/filename
%		{#4} % label
%		{#5} % short caption
%		{#6} % caption
%}

% mail address with reduced margin for error
\newcommand{\mailto}[1]{\mbox{\href{mailto:#1}{\texttt{#1}}}} % teletype style makes it look similar to \url{}

% strikes a line through the argument text
\newlength{\strikeOutLen} % to save length of line
\newcommand{\strikeOut}[1]{
	\settowidth{\strikeOutLen}{#1} % set length of line
	\mbox{#1\protect\hspace{-\strikeOutLen}\rule[0.5ex]{\strikeOutLen}{0.1ex}}
}

% full line \framebox for comments etc
\newcommand{\boxed}[1]{
	\par
	\noindent
	\framebox[\linewidth]{
		\addtolength{\linewidth}{-5mm}
		\parbox{\linewidth}{#1}
		\addtolength{\linewidth}{5mm}
	}
	\par
}

% universal box used for warnings, todos, etc
\newcommand{\universalBox}[2]{\boxed{\textbf{#1:}\begin{center}\texttt{\large{#2}}\end{center}}}
% some sort of todo box
\newcommand{\todo}[1]{\universalBox{ToDo}{#1}}
% some sort of warning box
\newcommand{\warn}[1]{\universalBox{Warning}{#1}}



%%%%%%%%%%%%%%%%%%%%%%%%%%%%%% document specific %%%%%%%%%%%%%%%%%%%%%%%%%%%%%%

% define the author, title etc
\globalTitle{Development Guide}
\globalSubject{setlX \input{version}}
\globalAuthor{Herrmann, Tom}
\globalDate{\today}
\globalKeywords{}

% frequently used configuration options
\setboolean{linkHighlighting}{true}

%%%%%%%%%%%%%%%%%%%%%%%%%%%%%%%%%%%% setup %%%%%%%%%%%%%%%%%%%%%%%%%%%%%%%%%%%%

% document layout options
\KOMAoptions{
	listof=totoc, % add list of figures, list of listings etc to toc
	bibliography=totoc, % add bibliography to toc
	abstract=true, % use special formating for abstracts
	parskip=half % use space between paragraphs, don't indent the first line.
}
\areaset[5mm]{425pt}{675pt}

% enable or disable all indication of links

\hypersetup{
	colorlinks=true, % set all link colors to blue
	linkcolor=blue,
	citecolor=red,
	pdfborder={0 0 0} % make sure border around links stays disabled
}

% default settings for listings 
\lstset{
	frame=lines, % frame around the code
	basicstyle=\small\ttfamily, % font size
	numbers=left, % enable line numbering
	numberstyle=\tiny, % font size for numbering
	tabsize=4, % size of tabs in spaces
	breaklines=false, % no line wrap
	xleftmargin=2em, % margin between left page border and listing
	numbersep=1.5em, % margin between numbering and code
	captionpos=b, % caption below the listing
	extendedchars=false % disable special char handling
	showstringspaces=false, % do not show special spaces character in strings
}

%%% format aliases for reformation in a context sensitive way

% command example
\newcommand{\command}[1]{\texttt{#1}}

% parameter definition
\newcommand{\param}[2]{
	\item[] \command{#1}\\
            \begin{tabular}{ l p{30 em} }
                   \hspace*{1 em} & #2 \\
            \end{tabular}
}

% folder description
\newcommand{\folder}[2]{
	\item \command{#1}\\
            \begin{tabular}{ l p{30 em} }
                   \hspace*{1 em} & #2 \\
            \end{tabular}
}

% setlX and SetlX
\def\setlX{\textsc{setlX}}    % setlX --- the interpreter
\def\SetlX{\textsc{SetlX}}    % SetlX --- the language
\def\Setl{\textsc{Setl}}      % Setl  --- the language
\def\SetlTwo{\textsc{Setl2}}  % Setl2 --- the language



%%%%%%%%%%%%%%%%%%%%%%%%%%%%%%%%%%%%%%%%%%%%%%%%%%%%%%%%%%%%%%%%%%%%%%%%%%%%%%%
%%%%%%%%%%%%%%%%%%%%%%%%%%%%%%%% document text %%%%%%%%%%%%%%%%%%%%%%%%%%%%%%%%
%%%%%%%%%%%%%%%%%%%%%%%%%%%%%%%%%%%%%%%%%%%%%%%%%%%%%%%%%%%%%%%%%%%%%%%%%%%%%%%

\begin{document}
\begin{titlepage}
\maketitle

\vfill

\begin{center}
\Large
``If you have any trouble sounding condescending,\\
find a Unix user to show you how it's done.''\\
Scott Adams
\end{center}

\vfill
\end{titlepage}

\tableofcontents

\newpage

\section{Overview}

This is the development guide for \setlX, an interpreter for the \SetlX{} (\underline{set} \underline{l}anguage e\underline{x}tended) programming-language.

The \setlX{} interpreter, which currently is the reference implementation for the \SetlX{} language, was implemented by Tom Herrmann.\\
Its official homepage is \url{http://setlX.randoom.org/}.

I would like to encourage you to send your bug-reports, code-changes and\slash{}or questions about the \setlX{} interpreter via e-mail to \mailto{setlx@randoom.org}.

\section{General Concepts}

This interpreter is implemented in Java and uses object oriented approaches, where possible. All file and variable names are chosen to reflect their purpose in the program. Complex algorithms should be documented where implemented, but when this implementation is scattered into multiple classes they are also documented in section \ref{specificConcepts}.

Note that this guide assumes extensive knowledge and understanding of the \SetlX{} programming language itself.

\subsection{Folder Structure}

The following folders are used to structure the interpreter and its surrounding components:

\begin{itemize}
	\folder{documentation}
			{\LaTeX{} sources for the documentation.}
	\folder{example\_SetlX\_code}
			{Example programs in \SetlX{} to test the interpreter and demonstrate some of its functions.\\&
			 (Also includes some programs, which where automatically converted from \SetlTwo{} sources.)}
	\folder{grammar\_pure}
			{\SetlX{} grammar in Antlr syntax as used by the interpreter, but without any Java statements. This `pure' version can be automatically created from the interpreter syntax with the included `EBNF\_extractor' by executing \command{make update}. Also includes a simple program to parse \SetlX{} code without interpretation.}
	\folder{src}
			{Launching script, make files and source code of the interpreter itself.}
	\begin{itemize}
		\folder{antlr}
				{Antlr `jar' file and license.}
		\folder{grammar}
				{\SetlX{} grammar in Antlr syntax, which is used to generate Java files.}
		\folder{java-src}
				{Java code of the interpreter (see section \ref{srcStructure}).}
		\folder{simpleTest}
				{Some very simple \SetlX{} code examples to test specific features for regressions, which are otherwise hard to detect. This test can be run by executing `\command{make test}'.}
	\end{itemize}
\end{itemize}

\subsection{Source Code Structure}\label{srcStructure}

The code is structured into multiple packages. The interpreter packages are somewhat modeled after the common names which describe elements of imperative programming languages:

\begin{itemize}
	\folder{org.randoom.setlx.boolExpressions}
			{Boolean expressions}
	\folder{org.randoom.setlx.exceptions}
			{Exceptions which handle the various errors which could occur.}
	\folder{org.randoom.setlx.expressions}
			{Non-boolean expressions}
	\folder{org.randoom.setlx.functions}
			{Predefined functions}
	\folder{org.randoom.setlx.grammar}
			{Java file generated by Antlr}
	\folder{org.randoom.setlx.statements}
			{Statements}
	\folder{org.randoom.setlx.types}
			{Value-types used by the interpreter.}
	\folder{org.randoom.setlx.utilities}
			{Various utility classes.}
\end{itemize}

\section{Specific Concepts}\label{specificConcepts}

This section describes specific implementations, which are scattered into multiple classes and therefore lack a dedicated place to describe their interactions.

\subsection{Predefined Functions}

All predefined functions in \setlX{}, like `\command{read()}' and `\command{isSet(x)}', are dynamically loaded at runtime on their first invocation using Java's reflection API. This makes it relatively easy to create new functionality without the need of changing (or even completely understanding) the rest of the interpreter.

These functions are dynamically loaded by the `VariableScope' class\\ (\command{org/randoom/setlx/utilities/VariableScope.java}):

When resolving an identifier the current and global scopes are searched. If no matching value is found, the predefined functions are searched for the same identifier. If still no value can be retrieved, Java's Math functions are queried.

In order to be found by this algorithm, predefined functions have to follow strict guidelines:

\begin{itemize}
	\item They need to be placed in the `\command{org.randoom.setlx.functions}' package.
	\item Each functions needs a separate class named `\command{PD\_<name>}', where \command{<name>} is the name of the function in the exact case and spelling as should be used in the interpreter.\\
		E.g. the `\command{isSet()}' function is defined in a class called `\command{PD\_isSet}'.
	\item The class needs to inherit from `\command{PreDefinedFunction}' in the same package.
	\item The `\command{super("name");}' constructor must be called, where \command{"name"} is the name of the function.
	\item The following variable must be present in the class and set with the invocation of the classes constructor (the constructor itself should be private):
\begin{lstlisting}[frame=none,numbers=none]
public final static PreDefinedFunction DEFINITION
                                    = new PD_<name>();
\end{lstlisting}
	\item And finally the actual work of the function has to be implemented inside the `\command{execute()}' method. The value returned by this method is used as return value of the function during execution. If no value should be returned, return `\command{Om.OM}' instead (import \command{org.randoom.setlx.types.Om}).

\end{itemize}

A simple predefined function without parameters is shown in figure \ref{lblPDFunction}.

\includeListing[e]{../src/java-src/org/randoom/setlx/functions/PD_now.java}{language=java}{lblPDFunction}{}{Predefined Function `\command{now()}'}

\nonSubsubsection{Using Parameters in Predefined Functions}

If the predefined function uses parameters, they have to be defined in the functions constructor by invoking:
\begin{lstlisting}[frame=none,numbers=none]
addParameter("value");
\end{lstlisting}

Hereby \command{"value"} is the name of the parameter. The name of the parameter is only shown when the function itself is used as an argument to print() inside a \SetlX{} program.

Read-Write parameters (`\command{rw}') can be defined using:
\begin{lstlisting}[frame=none,numbers=none]
addParameter("value", ParameterDef.READ_WRITE);
\end{lstlisting}

When `\command{execute(List<Value> args, List<Value> writeBackVars)}' is called, the current values of these parameters are placed inside the `\command{args}' list in \emph{the same order as they are added in the constructor}. When the predefined function is called with a different number of parameters from within \SetlX{}, the user gets an error and the execution is halted. If the predefined function should be allowed to handle more or fewer parameters as defined, the following methods can be called inside the constructor after parameter definitions:
\begin{lstlisting}[frame=none,numbers=none]
enableUnlimitedParameters();
allowFewerParameters();
\end{lstlisting}

To (re)set the value of read-write parameters, the following function has to be called inside `\command{execute()}':
\begin{lstlisting}[frame=none,numbers=none]
writeBackVars.add(new SetlReal("1.0"));
\end{lstlisting}

These have to occur in \emph{the same order and number} as read-write parameters are defined in the constructor. A predefined function using a single read-write parameter is shown in figure \ref{lblPDFunctionRW}.

\includeListing[e]{../src/java-src/org/randoom/setlx/functions/PD_from.java}{language=java}{lblPDFunctionRW}{}{Predefined Function `\command{from()}' using a read-write parameter}

\nonSubsubsection{Prevent Introduction of new Scope}

For some functions it is desirable to execute then without introducing a new scope.
To this effect, the following method can be inside the constructor after parameter definitions:
\begin{lstlisting}[frame=none,numbers=none]
doNotChangeScope();
\end{lstlisting}

\nonSubsubsection{Predefined Math Functions}

When an identifier can not be found, after searching for predefined functions, a query for a matching Java Math method, with a single double as parameter, is performed. When found, the matching method is wrapped inside the `MathFunction' class\\ (\command{org/randoom/setlx/functions/MathFunction.java}). This class behaves as any other predefined function and converts the \setlX{} value type `Real' into a standard java double, which can used with Java's Math functions.

\section{Build Targets}

The interpreter can be intelligently (re)build using make. Even though the standard java compiler can resolve and compile dependencies, the rather complex make file is required for partial (re)builds, as files are not rebuild by this compiler when one of their dependencies was updated.

Make targets:

\begin{itemize}
	\item Standard build of the interpreter, which creates a self-contained `.jar' file:
\begin{lstlisting}[frame=none,numbers=none]
make
\end{lstlisting}

	\item Build a version without creating a jar file:

\begin{lstlisting}[frame=none,numbers=none]
make build
\end{lstlisting}

	\item Build a current version and execute the interactive mode:

\begin{lstlisting}[frame=none,numbers=none]
make interactive
\end{lstlisting}

	\item Build a current version from source and execute a small set of regression tests:

\begin{lstlisting}[frame=none,numbers=none]
make test
\end{lstlisting}

	\item Delete all automatically created files, except the `.jar' file:

\begin{lstlisting}[frame=none,numbers=none]
make clean
\end{lstlisting}

	\item Delete all created files:

\begin{lstlisting}[frame=none,numbers=none]
make dist-clean
\end{lstlisting}

\end{itemize}

\section{Semi Automatic Testing}\label{testing}

As a sorry excuse for quality assurance and control a simple \command{test\_all} script is included with the interpreters source. This script should be run after every functional change, but at least before creating a new version, by executing

\begin{lstlisting}[frame=none,numbers=none]
./test_all
\end{lstlisting}

The logic of the script is as follows:
\begin{itemize}
	\item First \command{make test} is executed and its result verified.
	\item Each source file ending in \command{.stlx} inside the \command{example\_SetX\_code} folder will be executed, if a file with the same basename, but ending in \command{.stlx.reference} exists.
	\item If the executed program requires user input, a file with the same basename, but ending in \command{.stlx.input} must be created, including an input per line. This fill will be piped into the interpreter when executing the source file.
	\item Execution uses the \command{src} folder as working directory and starts \setlX{} with the \command{--predictableRandom} option, thus ensuring predictable results. 
	\item The output (on both stdout and stderr) of the execution and the reference file are compared using diff.
	\item When the resulting diff is empty, the next source file is executed.
	\item Should the resulting diff file not be empty, the script stops. The diff result should then be examined.
\end{itemize}

\section{Undocumented Features}

In addition to the options documented in the manual, \setlX{} has a few `undocumented' options:

\begin{itemize}
	\param{--dump <file-name>}
	      {Writes loaded code into a file (has no effect in interactive mode).}
	\param{--unhideExceptions}
	      {Exceptions, which may occur unexpectedly in the Java code (e.g. \command{NullPointerException}), are hidden from the user and instead display a generic error message. When setting this option, the interpreter will print the usual Java stack trace of all unexpected exceptions.}
\end{itemize}

\section{Distribution}

Two different `distributions' can be automatically created (on an Unix-like OS):

\begin{enumerate}
	\item A binary only distribution, which should work on all Unix-like OS and Windows \emph{without} the need of the full Java JDK. Any Java Runtime (JRE) installation which is newer or the same major version as the JDK which created the distribution should work. This distribution does \emph{not} include any \SetlX{} code.

	\item The development kit, which includes everything used in the development of the interpreter itself: the full interpreter source, various \SetlX{} example programs, the grammar of the interpreter in `pure' form and the \LaTeX{} sources of the documentation.
\end{enumerate}

\subsection{Building distributable zip-files}

The automated distribution creation script can be launched by executing

\begin{lstlisting}[frame=none,numbers=none]
./createDistributions
\end{lstlisting}

Note that this process will `clean' the source (e.g. remove all automatically created files), rebuild the interpreter and its documentation and finally cleans the source again. It is advisable to run the \command{test\_all} script before creating a new version (see section \ref{testing}).

\subsection{Distributing setlX}

You may distribute \setlX{} in all three above mentioned version, or any other version, to anyone and for any purpose, and make changes to it, even without permission from the author or notice to the author, as long as you do not remove any copyright information, and also comply with the antlr license (see \command{src/antlr/antlr\_LICENSE.txt}).

However, I would like to encourage you to provide the full development kit to anyone you distribute one of the other versions to, at least as an option and\slash{}or on request.

\section{Used Software}

The following software was used when developing the \setlX{} interpreter:

\begin{itemize}
	\item OpenJDK 1.7.0\_b147, compatible to Oracles JDK 1.7 (aka JDK 7)
	\item Antlr 3.4, which is shipped as `.jar' with the interpreter
	\item \LaTeX{} and various \LaTeX{} packages:
		\begin{itemize}
			\item KOMA script (scrartcl)
			\item ifthen
%			\item eurosym
			\item ucs
			\item inputenc
%			\item babel
			\item listings
%			\item graphicx
			\item hyperref
			\item hypcap
		\end{itemize}
	\item GNU Make 3.82
	\item Zip 3.0
	\item Fedora Linux 16 (64 Bit)
\end{itemize}

\end{document}
