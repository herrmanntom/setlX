\def\myDocumentTypeArticle{-}
\InputIfFileExists{\jobname_header}{}{\input{config/header_V001}}

%%%%%%%%%%%%%%%%%%%%%%%%%%%%%% document specific %%%%%%%%%%%%%%%%%%%%%%%%%%%%%%

% define the author, title etc
\globalTitle{Development Guide}
\globalSubject{setlX \input{version}}
\globalAuthor{Herrmann, Tom}
\globalDate{\today}
\globalKeywords{readme}

% frequently used configuration options
\setboolean{linkHighlighting}{true}

%%%%%%%%%%%%%%%%%%%%%%%%%%%%%%%%%%%% setup %%%%%%%%%%%%%%%%%%%%%%%%%%%%%%%%%%%%

\InputIfFileExists{\jobname_configuration}{}{\input{config/configuration_V001}}

%%%%%%%%%%%%%%%%%%%%%%%%%%%%%%%%%%%%%%%%%%%%%%%%%%%%%%%%%%%%%%%%%%%%%%%%%%%%%%%
%%%%%%%%%%%%%%%%%%%%%%%%%%%%%%%% document text %%%%%%%%%%%%%%%%%%%%%%%%%%%%%%%%
%%%%%%%%%%%%%%%%%%%%%%%%%%%%%%%%%%%%%%%%%%%%%%%%%%%%%%%%%%%%%%%%%%%%%%%%%%%%%%%

\begin{document}
\begin{titlepage}
\maketitle
\vfill
\tableofcontents
\end{titlepage}

\section{Overview}

This is the development guide for \setlX, an interpreter for the \SetlX{} programming-language.

The \setlX{} interpreter, which currently is the reference implementation for the \SetlX{} language, was implemented by Tom Herrmann.

I would like to encourage you to send your bug-reports, code-changes and or questions about the \setlX{} interpreter via e-mail to \mailto{setlx@randoom.org}.

\section{Architecture and General Concepts}

tbd

\nonSubsection{Source Code Structure}

\begin{itemize}
	\param{boolExpressions}
			{tbd}
	\param{exceptions}
			{tbd}
	\param{expressions}
			{tbd}
	\param{functions}
			{tbd}
	\param{statements}
			{tbd}
	\param{types}
			{tbd}
	\param{utilities}
			{tbd}
\end{itemize}

\section{Specific Concepts}

tbd

\subsection{Environment (Stores Variables in Current Scope)}

tbd

\subsection{Dynamically Loaded Functions}

tbd

\subsection{Iterators}

tbd

\section{Distribution}

Three different `distributions' can be automatically created (on an Unix-like OS):

\begin{enumerate}
	\item A binary only distribution, which should work on all Unix-like OS and Windows without the need of the full Java JDK. Any Java Runtime (JRE) installation which is newer or the same major version as the JDK which created the distribution should work. This distribution does not include any \SetlX{} code.

	\item A source distribution, which must be rebuild by the user, which is automatically attempted on first launch. A Java JDK has to be installed for this to work. The version of the users JDK can be lower or even incompatible to the JDK which created the distribution. This distribution also does not include \SetlX{} code, except a simple program to test the interpreter (see `make test' in the manual).

	\item The development kit, which includes everything used in the development of the interpreter itself. This distribution includes various \SetlX{} example programs and the grammar of the interpreter in both `EBNF' and `pure' form.
\end{enumerate}

\subsection{Building distributable zip-files}

The automated distribution creation script can be launched by executing

\begin{lstlisting}
./createDistributions
\end{lstlisting}

Note that this process will `clean' the source (e.g. remove all automatically created files), rebuild the interpreter and all documentation and clean the source again.

\subsection{Distributing setlX}

You may distribute \setlX{} in all three above mentioned version, or any other version, to anyone and for any purpose, and make changed to it, without permission from the author or notice to the author, as long as you do not remove any copyright information, and also comply to the antlr license (see \command{interpreter/antlr/antlr\_LICENSE.txt}).

However, I would like to encourage you to provide the full development kit to anyone you distribute one of the other versions to, at least as an option.

\section{Used Software}

The following software was used when developing the \setlX{} interpreter:

\begin{itemize}
	\item OpenJDK 1.6.0\_22, compatible to Oracles JDK 1.6 (aka version 6)
	\item Antlr 3.3, which is shipped as `.jar' with the interpreter
	\item pdfTeX, latex and various latex packages
	\item GNU Make 3.82
	\item Zip 3.0
	\item Fedora Linux 15 (64 Bit)
\end{itemize}

\end{document}
